%=============================================================================
% Abstract
% Copyright (c) 2018. Lester James V. Miranda
%
% This file is part of thesis-manuscript.
%
% thesis-mansucript is free software: you can redistribute it and/or modify
% it under the terms of the GNU General Public License as published by
% the Free Software Foundation, either version 3 of the License, or
% (at your option) any later version.
%
% thesis-manuscript is distributed in the hope that it will be useful,
% but WITHOUT ANY WARRANTY; without even the implied warranty of
% MERCHANTABILITY or FITNESS FOR A PARTICULAR PURPOSE.  See the
% GNU General Public License for more details.
%
% You should have received a copy of the GNU General Public License
% along with thesis-manuscript.  If not, see <http://www.gnu.org/licenses/>.
%
% Created by: Lester James V. Miranda <ljvmiranda@gmail.com>
%=============================================================================

\par Protein function prediction is a fundamental task with applications in
medicine and healthcare. However, the cost and slow-pace of biochemical
techniques hinder this process, causing a major backlog to the number of
unannotated proteins we have today. Machine learning techniques are suitable
for this data-intensive task, but they only work well given good data
representations. Thus, it is important to ensure that our features are
meaningful and relevant to the classifier. 

\par In this work, we introduce two autoencoder-based techniques to extract
task-relevant representations for better protein classification. We first
explored a stacked denoising autoencoder, commonly-used in images, and examined
its efficacy in the protein domain. Then, we designed a mutually-competitive
autoencoder that extracts task-relevant features by letting the neurons compete
with one another. The features derived by these autoencoders were fed to a
binary-relevance support-vector machine for classification.  We tested both
models on two protein benchmarks, investigating hyperparameter influence and
measuring model quality, then we compared its performance against other
techniques in literature.

\par Results show that our proposed models perform better than other methods,
suggesting that extracting \emph{relevant} features, not mere feature
extraction, is crucial. Moreover, both models outperformed the baseline, i.e. a
model without feature extraction, indicating that it is important to derive new
features than using a protein's raw attributes.  Lastly, the
mutually-competitive autoencoder outperforms the stacked denoising
autoencoder\textemdash exhibiting not only a finer control during extraction,
but improved classifier performance.  Overall, our work demonstrated the
benefit of using these autoencoder-based methods to extract task-relevant
features for better protein classification.
