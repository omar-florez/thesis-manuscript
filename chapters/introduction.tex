%=============================================================================
% Introduction
% Copyright (c) 2018. Lester James V. Miranda
%
% This file is part of thesis-manuscript.
%
% thesis-mansucript is free software: you can redistribute it and/or modify
% it under the terms of the GNU General Public License as published by
% the Free Software Foundation, either version 3 of the License, or
% (at your option) any later version.
%
% thesis-manuscript is distributed in the hope that it will be useful,
% but WITHOUT ANY WARRANTY; without even the implied warranty of
% MERCHANTABILITY or FITNESS FOR A PARTICULAR PURPOSE.  See the
% GNU General Public License for more details.
%
% You should have received a copy of the GNU General Public License
% along with thesis-manuscript.  If not, see <http://www.gnu.org/licenses/>.
%
% Created by: Lester James V. Miranda <ljvmiranda@gmail.com>
%=============================================================================

\chapter{Introduction}
\label{Introduction}

\par In bioinformatics, predicting a protein's function is a fundamental task.
If a protein's functional category is known, then designing drugs or
identifying a disease's molecular mechanism is achievable. However, acquiring
this knowledge is slow and difficult due to the complexity in a protein's
biological information, and the unfamiliarity of life's organization at the
molecular level (\cite{baldi2001bioinformatics}). In addition, more and more
proteins are being characterized by high-throughput sequencing techniques
today, creating vast stores of genomic data available for use
(\cite{gaudet2017gene, cozzetto2017computational}). This widens the gap between
annotated and unannotated proteins, urging the need for a more efficient,
accurate, and fast set of prediction techniques in the presence of
``genomic big data.''

\par Machine learning techniques have been observed to perform well in complex
tasks involving large amounts of data (\cite{chen2014data}). They have been
exceptional in recognizing patterns and learning from a vast number of examples 
(\cite{lecun2015deep}). Before prediction, a machine learning model must first
discern a pattern from a set of data, and use that to infer the category, or 
\textit{class}, a new sample belongs to. Because most proteins can perform
multiple functions at once, \textit{multi-label classification}\textemdash
a subset of machine learning\textemdash is used. However, the effectiveness of
a multi-label classification model still depends on the pattern it learns, and
how well it represents the input data provided.

\par Learning to understand data and representing it in a form beneficial for a
machine learning model is known as \textit{representation learning}. Although
it is entirely possible to manually engineer characteristics, or \textit{features}
, of a dataset, it is labor-intensive and requires thorough domain-expertise 
(\cite{bengio2013representation}).  Instead, it is preferable to automatically
extract useful information from data using a \textit{feature extractor}, and
ensure that the new features are relevant. Selecting more relevant features is
apt for protein datasets because it is noisy, i.e., both relevant and irrelevant
data are present. Properly discerning which characteristics are important and 
transforming them into features useful to a predictor is essential in
identifying protein functions.

\par This research presents a novel architecture based on the autoencoder neural
network. The proposed network motivates selective behavior in order to produce
relevant features. This is done through mutual competition: adjusting the
backpropagation path to keep a select number of neurons active while ensuring
meaningful representations. It is expected that this method will be able to
predict protein functions well, as compared to using raw data directly or to
other techniques in literature. 

\newpage

\par \noindent The rest of the document is organized as follows:

\begin{itemize} 
    \item The remaining sections of Chapter \ref{Introduction}
        discuss the context behind this work (Sec. \ref{Background}), the
        research motivation (Sec. \ref{Motivation}), and the formulation of the
        research problem (Sec. \ref{Problem}).  
    \item Chapter \ref{LiteratureReview} comments on the related literature
        that also tackles the problem of protein function prediction. 
    \item Chapter \ref{Methodology} outlines the proposed method
        and gives an overview of its training and testing schemes (Sec. 
        \ref{Methodology}). In addition, an explanation of the datasets used (Sec.
        \ref{Datasets}) and a description of the experimental environment
        (Sec. \ref{ExperimentalEnvironment}) will be done. 
    \item Chapter \ref{Results} discusses the experiments conducted to evaluate the 
        proposed method: benchmarking performance with other techniques 
        (Sec. \ref{Benchmarking}), measuring model quality (Sec.\ref{ModelQuality}),
        and estimating feature relevance (Sec.\ref{FeatureRelevance}).
    \item Lastly, Chapter \ref{Conclusion} concludes the work. Potential avenues for 
        future research (Sec. \ref{FutureWork}) will be discussed.
\end{itemize}
