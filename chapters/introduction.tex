%=============================================================================
% Introduction
% Copyright (c) 2018. Lester James V. Miranda
%
% This file is part of thesis-manuscript.
%
% thesis-mansucript is free software: you can redistribute it and/or modify
% it under the terms of the GNU General Public License as published by
% the Free Software Foundation, either version 3 of the License, or
% (at your option) any later version.
%
% thesis-manuscript is distributed in the hope that it will be useful,
% but WITHOUT ANY WARRANTY; without even the implied warranty of
% MERCHANTABILITY or FITNESS FOR A PARTICULAR PURPOSE.  See the
% GNU General Public License for more details.
%
% You should have received a copy of the GNU General Public License
% along with thesis-manuscript.  If not, see <http://www.gnu.org/licenses/>.
%
% Created by: Lester James V. Miranda <ljvmiranda@gmail.com>
%=============================================================================

\chapter{Introduction}
\label{Introduction}

\par In bioinformatics, predicting a protein's function is a fundamental
task. This knowledge can lead us to various applications such as drug design
or disease identification (\cite{baldi2001bioinformatics}). However,
acquiring this information is difficult, for existing biochemical techniques
are slow and expensive (\cite{cozzetto2017computational}). As the number of
proteins being discovered increases, the backlog of unannotated protein data
piles up (\cite{gaudet2017gene}). This then calls for a fast, accurate, and
efficient set of prediction techniques in the face of ``genomic big data.''
  
\par Machine learning has shown promise in complex tasks involving large
amounts of data (\cite{chen2014data}). They have been exceptional in
recognizing patterns and learning from a vast number of samples
(\cite{lecun2015deep}). Before prediction, a machine learning model must
discern a pattern from a dataset, and use that to infer the category, or
\textit{class}, a new sample belongs to. Because most proteins can perform
multiple functions at once, we frame the protein function prediction problem
as a multilabel classification task. In this regard, our primary question is:

\begin{quote}
    \itshape
    \small
    Given a set of protein characteristics, what can we infer about
    its function/s?
\end{quote}

\noindent It is important to note that the effectiveness of a classification
model strongly depends on the pattern it learns, and how well it represents
the input data provided.

\par Feature extraction aims to represent raw data into a form beneficial to
a machine learning model. Although it is entirely possible to manually
engineer characteristics, or \textit{features}, from a dataset, it is
labor-intensive and requires thorough domain-expertise
(\cite{bengio2013representation}). Instead, it is preferable to automatically
derive useful information from our inputs and ensure that the new features
are relevant. The better a feature extraction method can represent raw data,
the easier a classification model can discriminate between its samples. For
protein datasets, extracting \textit{relevant} information is challenging for
two reasons: protein samples are (1) noisy, and (2) high-dimensional. Noise
introduces irrelevant information to the classification task. It is inherent
in data-acquisition and intrinsic to biological complexity, On the other
hand, high-dimensional datasets suffer from the \textit{curse of
dimensionality}, further complicating the discrimination process.
Distinguishing which characteristics are important and transforming them into
features useful to a predictor is essential in identifying protein functions.

\newpage

\par This work's overarching theme is about the effect of \textit{extracting
relevant features} on the performance of a multilabel classification model in
predicting protein functions. We put emphasis on feature relevance, given the
noisy and high-dimensional nature of protein data. We hypothesize that by
obtaining relevant features, we can achieve better performance than naively
extracting features or not extracting them at all. The models that we will
introduce in this work were based from an autoencoder neural network. For the
time being, know that autoencoders learn new representations by actively
reconstructing the input with the presence of information bottlenecks. We
studied two autoencoder architectures, and tested them on protein benchmarks.
The main contributions of this research are as follows:

\begin{itemize}
    \item We applied a stacked denoising autoencoder, a technique
    commonly-used in image denoising, to the protein function prediction problem.
    \begin{quote}
    \par We have demonstrated the efficacy of the autoencoder in a different
    problem domain by learning \textit{robust} features to aid multilabel
    classification. However, autoencoder training is inherently greedy, and
    motivating the network to learn sparse yet relevant features should
    offset this problem.
    \end{quote}
    \item We designed a mutually-competitive autoencoder architecture that
    motivates the creation of sparse yet relevant features from raw data.
    \begin{quote}
    \par We have proven that by learning \textit{relevant} representations
    of the raw inputs, classification performance can improve. This architecture
    outperformed not only a model without feature extraction, but also other
    feature extraction methods by a significant margin. 
    \end{quote}
\end{itemize}



\par \noindent This document is divided into five (5) chapters, with the rest
organized as follows:

\begin{itemize}
    \item \textsc{Chapter \ref{BackgroundChapter}} \textit{(Background of the
    Study)} provides an overview of the protein function prediction problem,
    the idea behind feature extraction, and the formulation of a multilabel
    classification problem. It also reviews related works that will be
    benchmarked against our proposed method.
    \item \textsc{Chapter \ref{SDAEChapter}} \textit{(Feature Extraction
    using a Stacked Denoising Autoencoder for Protein Function Prediction)}
    investigates the use of denoising autoencoders in the protein function
    prediction problem in a multilabel setting.
    \item \textsc{Chapter \ref{SelectiveChapter}} \textit{(Selective Feature
    Extraction using a Mutually-Competitive Autoencoder)} describes our proposed
    autoencoder architecture and key experiments that examine its performance
    on a protein function prediction task.
    \item \textsc{Chapter \ref{ConclusionsChapter}} \textit{(Conclusions)} gives
    a concise summary of this research, key insights, and potential avenues for
    future research.
\end{itemize}