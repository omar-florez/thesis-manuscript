\chapter{Results and Discussion}
\label{Results}


\section{Comparison to other models}
\label{Benchmarking}

\section{Measuring model quality}
\label{ModelQuality}
\par In this section, the proposed model will be compared to other works in
literature. The existing works follow the same pipeline for protein function
prediction\textemdash extract features from data, then classify with the
extracted features:

\begin{itemize}
    \item \cite{wang2013protein} applied principal component
    analysis (PCA) to reduce the dimensions of protein data from $353$ to
    $204$. 
    \item \cite{chicco2014deep} used a deep autoencoder with two-layers
    to predict the protein functions of \textit{B.taurus} and \textit{G.
    gallus}. 
    \item \cite{miranda2017feature} implemented a stacked denoising
    autoencoder to obtain robust features from raw data to predict protein
    functions.  
\end{itemize}

\par In addition, we will also compare against a baseline model without feature
extraction. All features, extracted or not, will be trained on a
binary-relevance SVM for classification (using both gaussian and linear
kernels). The pipeline will run for ten (10) trials with the mean and
standard deviation of the results on the test data reported. 

\begin{table}[!ht]
    \footnotesize
    \centering
    \caption{Results for the Yeast Dataset using SVM with Gaussian Kernel}
    \label{results:yeast_svm}
    \begin{threeparttable}
    \begin{tabular}{@{}rrlllll@{}}
    \toprule
    && \multicolumn{5}{c}{Prediction Model} \\ \cmidrule{3-7}
    \multicolumn{2}{r}{Metrics}               & Baseline                & Wang, 2013 & Chicco, 2014 & Miranda, 2017 & Proposed            \\ \midrule
\multicolumn{2}{l}{Area ROC} \\
                    & \textit{micro}            & $0.6668 \pm 0.0031$ & $0.6559 \pm 0.0020$ & $0.6430 \pm 0.0012$ & $0.6679 \pm 0.0025$ & $\mathbf{0.7320 \pm 0.0021} $ \\
                     & \textit{macro}            & $0.6548 \pm 0.0031$ & $0.6590 \pm 0.0024$ & $\mathbf{0.6624 \pm 0.0011}$ & $0.6484 \pm 0.0022$ & $0.6455 \pm 0.0018 $ \\
                     & \textit{samples}            & $0.6700 \pm 0.0032$ & $0.6559 \pm 0.0019$ & $0.6443 \pm 0.0011$ & $0.6569 \pm 0.0026$ & $\mathbf{0.7433 \pm 0.0040} $ \\
\multicolumn{2}{l}{F-score} \\
                    & \textit{micro}            & $0.5483 \pm 0.0041$ & $0.5381 \pm 0.0023$ & $0.5304 \pm 0.0014$ & $0.5792 \pm 0.0028$ & $\mathbf{0.6289 \pm 0.0024} $ \\
                     & \textit{macro}            & $0.5970 \pm 0.0035$ & $0.6033 \pm 0.0024$ & $0.5896 \pm 0.0011$ & $0.6132 \pm 0.0023$ & $\mathbf{0.6299 \pm 0.0024} $ \\
                     & \textit{samples}            & $0.5359 \pm 0.0032$ & $0.5242 \pm 0.0022$ & $0.5171 \pm 0.0014$ & $0.5720 \pm 0.0034$ & $\mathbf{0.6086 \pm 0.0038} $ \\ 
\multicolumn{2}{l}{Hamm. Loss}            & $0.3222 \pm 0.0021$ & $0.3430 \pm 0.0018$ & $0.3535 \pm 0.0011$ & $0.2310 \pm 0.0031$ & $\mathbf{0.2241 \pm 0.0019}$ \\
\bottomrule
    \end{tabular}
    \begin{tablenotes}
        \item SVM Hyperparameters for Proposed Model: $C=1.000$, $\gamma=1.67e-3$
    \end{tablenotes}
    \end{threeparttable}
\end{table}


\section{Estimating feature relevance}
\label{FeatureRelevance}
