%=============================================================================
% Implementation Details Appendix
% Copyright (c) 2018. Lester James V. Miranda
%
% This file is part of thesis-manuscript.
%
% thesis-manuscript is free software: you can redistribute it and/or modify
% it under the terms of the GNU General Public License as published by
% the Free Software Foundation, either version 3 of the License, or
% (at your option) any later version.
%
% thesis-manuscript is distributed in the hope that it will be useful,
% but WITHOUT ANY WARRANTY; without even the implied warranty of
% MERCHANTABILITY or FITNESS FOR A PARTICULAR PURPOSE.  See the
% GNU General Public License for more details.
%
% You should have received a copy of the GNU General Public License
% along with thesis-manuscript.  If not, see <http://www.gnu.org/licenses/>.
%
% Created by: Lester James V. Miranda <ljvmiranda@gmail.com>
%=============================================================================

\chapter{Implementation Details}
\label{AppendixImplementation}

\par In this chapter, we'll list down necessary details regarding the
implementation of our experiments. Note that the ``$\star$'' symbol represents
a dependent variable. Lastly, we implemented grid search and a fine random
search to obtain optimal values for the BR-SVM classifier.

\section{Hyperparameters used on SdAE experiments}
\par The tables below correspond to the experiments listed in Section
\ref{SDSetup}. When testing the effect of various hyperparameters,
we kept the network and training time at a minimum.

\begin{table}[!h]
%
\centering
\caption{Hyperparameters for SdAE model characterization}
%
\begin{tabular}{@{}rr*{9}{l}@{}}
\toprule
      &  && \multicolumn{5}{c}{Feature Extractor} && \multicolumn{2}{c}{Classifier} \\ \cmidrule{4-8} \cmidrule{10-11}
Experiment     & Dataset && $e$     & $r$ & optimizer & epochs & reg.       && $C$       & $\gamma$  \\\midrule
Encoding units & Yeast   && $\star$ & 0.6 & Adam      & 200    & \num{1e-2} && \num{1.0} & \num{0.01} \\
               & Genbase && $\star$ & 0.6 & Adam      & 100    & \num{1e-2} && \num{1.0} & \num{0.02} \\
Noise rate     & Yeast   && 75  & $\star$ & Adam      & 200    & \num{1e-2} && \num{1.0} & \num{0.01} \\
               & Genbase && 50  & $\star$ & Adam      & 100    & \num{1e-2} && \num{1.0} & \num{0.02} \\\bottomrule
\end{tabular}
\end{table}



\par Here are the implementation details for the benchmarking part of the
experiments. To fully optimize the values, we conducted another round of grid
and random search for the BR-SVM hyperparameters, and a fine random search
for the noise rate based from the findings during characterization.

\begin{table}[!h]
%
\centering
\caption{Additional hyperparameters for SdAE model benchmarks}
%
\begin{tabular}{@{}r*{9}{l}@{}}
\toprule
        && \multicolumn{5}{c}{Feature Extractor} && \multicolumn{2}{c}{Classifier} \\ \cmidrule{3-7} \cmidrule{9-10}
Dataset && $r$        & optimizer & epochs & batch & reg.       && $C$         & $\gamma$       \\\midrule
Yeast   && $0.7451$   & Adam      & 1000   & 200   & \num{1e-2} && \num{1.00}  & \num{1.292e-2} \\
Genbase && $0.5988$   & Adam      & 750    & 100   & \num{1e-2} && \num{1.66e2}& \num{2.154e-4} \\\bottomrule
\end{tabular}
\end{table}

\begin{table}[!h]
    \centering
    \caption{Network architectures for SdAE model benchmarks}
    \begin{tabular}{@{}r*{4}{l}@{}}
    \toprule
    Dataset & Layer Name     & Shape        & Type          & Activation          \\ \midrule
    yeast   & inputLayer     & (None, 103)  & Input         & -                   \\
            & noiseLayer     & (None, 103)  & Noise         & -                   \\
            & encLayer1      & (None, 100)  & Dense         & Sigmoid             \\
            & encLayer2      & (None, 80)   & Dense         & Sigmoid             \\
            & decLayer1      & (None, 100)  & Dense-tied    & Linear              \\
            & decLayer2      & (None, 103)  & Dense-tied    & Linear              \\ \midrule
    genbase & inputLayer     & (None, 1186) & Input         & -                   \\
            & noiseLayer     & (None, 1186) & Noise         & -                   \\
            & encLayer       & (None, 50)   & Dense         & Linear              \\
            & decLayer       & (None, 1186) & Dense-tied    & Linear               \\ \bottomrule
    \end{tabular}
\end{table}



\newpage
\section{Hyperparameters used on MC experiments}
\par The tables below correspond to the experiments listed in Section
\ref{MCExperiments}. When testing the effect of various hyperparameters, we
kept the network and training time at a minimum. The default values during
characterization were obtained via a rough grid search.

\begin{table}[!h]
%
\centering
\begin{threeparttable}
\caption{Hyperparameters for MC model characterization}
%
\begin{tabular}{@{}rr*{8}{l}@{}}
\toprule
      &  && \multicolumn{4}{c}{Feature Extractor} &&
\multicolumn{2}{c}{Classifier} \\ \cmidrule{4-7} \cmidrule{9-10}
Experiment        & Dataset && $e$     & $k\%$ & $\alpha$ & epochs && $C$       & $\gamma$   \\\midrule
Encoding units    & Yeast   && $\star$ & 0.5   & 10.0     & 200    && \num{1.0} & $0.0017$   \\
                  & Genbase && $\star$ & 0.35  & 23.2     & 150    && \num{1.0} & $0.0023$   \\
Top-$k\%$ neurons & Yeast   && 75  & $\star$   & 10.0     & 200    && \num{1.0} & $0.0017$   \\
                  & Genbase && 150 & $\star$   & 23.2     & 150    && \num{1.0} & $0.0023$   \\
Competition amt.  & Yeast   && 75  &  0.5      & $\star$  & 200    && \num{1.0} & $0.0017$   \\
                  & Genbase && 150 &  0.35     & $\star$  & 150    && \num{1.0} & $0.0023$
\\\bottomrule
\end{tabular}
\begin{tablenotes}
\footnotesize
\item[1] We used the Adam optimizer for all models
\item[2] Regularization: \num{1e-2}, Learning rate: \num{0.01} 
\end{tablenotes}
\end{threeparttable}
\end{table}




\par A decision-tree was constructed to measure feature relevance.  The table
below describes the hyperparameters we used to build the tree.  Recall that the
number of encoding units is variable ($\star$), and we measured the
distribution for each value. For raw attributes, the features were directly
used to grow the tree.

\begin{table}[!h]
%
\centering
\caption{Hyperparameters for benchmarking feature relevance}
%
\begin{tabular}{@{}r*{9}{l}@{}}
\toprule
        && \multicolumn{3}{c}{Feature Extractor} && \multicolumn{4}{c}{Decision Tree} \\ \cmidrule{3-5} \cmidrule{7-10}
Dataset && $k$   & $\alpha$ & epochs && depth & l. rate & subsample & scale   \\\midrule
Yeast   && $0.6$ & $10.0$   & $1000$ && $7$   & $0.50$  & $0.90$    & $1.00$  \\
Genbase && $0.4$ & $10.0$   & $750$  && $9$   & $0.42$  & $0.90$    & $1.00$  \\\bottomrule
\end{tabular}
\end{table}



\par Below are the hyperparameters used for the benchmark analyses and
statistical comparison. 

\begin{table}[!h]
    \centering
    \caption{Network architectures for MC model benchmarks}
    \begin{tabular}{@{}r*{4}{l}@{}}
    \toprule
    Dataset & Layer Name     & Shape        & Type           & Activation          \\ \midrule
    yeast   & inputLayer     & (None, 103)  & Input          & -                   \\
            & encLayer       & (None, 500)  & Dense          & ReLU                \\
            & WTALayer       & (None, 500)  & Dense, $k=0.6$ & -                   \\
            & sparseLayer    & (None, 500)  & Dense, $\alpha=10.0$ & -             \\
            & decLayer1      & (None, 500)  & Dense-tied     & Linear              \\
            & decLayer2      & (None, 103)  & Dense-tied     & Linear              \\ \midrule
    genbase & inputLayer     & (None, 1186) & Input          & -                   \\
            & encLayer       & (None, 30)   & Dense          & ReLU                \\
            & WTALayer       & (None, 30)   & Dense, $k=0.6$ & -                   \\
            & sparseLayer    & (None, 30)   & Dense, $\alpha=5.32$ & -             \\
            & decLayer1       & (None, 30)   & Dense-tied     & Linear              \\
            & decLayer2       & (None, 1186) & Dense-tied     & Linear               \\ \bottomrule
    \end{tabular}
\end{table}

\begin{table}[!h]
%
\centering
\caption{Additional hyperparameters for MC model benchmarks}
%
\begin{tabular}{@{}r*{8}{l}@{}}
\toprule
        && \multicolumn{4}{c}{Feature Extractor} && \multicolumn{2}{c}{Classifier} \\ \cmidrule{3-6} \cmidrule{8-9}
Dataset &&  optimizer & epochs & batch & reg.       && $C$         & $\gamma$       \\\midrule
Yeast   &&  Adam      & 1000   & 200   & \num{1e-2} && \num{1.00}  & \num{1.63e-3} \\
Genbase &&  Adam      & 1000   & 100   & \num{1e-2} && \num{1.00}  & \num{1.24e-3} \\\bottomrule
\end{tabular}
\end{table}


\par Lastly, here are the hyperparameters used during the ablation tests. For
our architecture, setting $\{k=1, \alpha=0\}$ approximates a traditional
autoencoder while setting $\{k=\star, \alpha=0\}$ approximates a
winner-take-all autoencoder.

\begin{table}[!h]
%
\centering
\begin{threeparttable}
\caption{Hyperparameters for MC model ablation tests}
%
\begin{tabular}{@{}r*{8}{c}@{}}
\toprule
      && \multicolumn{4}{c}{Feature Extractor} &&
\multicolumn{2}{c}{Classifier} \\ \cmidrule{3-6} \cmidrule{8-9}
Dataset  && $e$  & $k\%$               & $\alpha$              & epochs && $C$       & $\gamma$       \\\midrule
 Yeast   && 500  & $\{1.0, 0.6, 0.6\}$ & $\{0.0, 0.0, 10.0\}$  & 1000   && \num{1.0} & \num{1.63e-3}  \\
 Genbase && 30   & $\{1.0, 0.6, 0.6\}$ & $\{0.0, 0.0, 5.32\}$  & 1000   && \num{1.0} & \num{1.24e-3}  \\\bottomrule
\end{tabular}
\begin{tablenotes}
\footnotesize
\item[1] The values inside the brackets were the ones used for each
step of the ablation test: $\{\text{AE}, \text{AE + WTA}, \text{AE + WTA + SL}\}$
\end{tablenotes}
\end{threeparttable}
\end{table}




