%=============================================================================
% Datasets Appendix
% Copyright (c) 2018. Lester James V. Miranda
%
% This file is part of thesis-manuscript.
%
% thesis-manuscript is free software: you can redistribute it and/or modify
% it under the terms of the GNU General Public License as published by
% the Free Software Foundation, either version 3 of the License, or
% (at your option) any later version.
%
% thesis-manuscript is distributed in the hope that it will be useful,
% but WITHOUT ANY WARRANTY; without even the implied warranty of
% MERCHANTABILITY or FITNESS FOR A PARTICULAR PURPOSE.  See the
% GNU General Public License for more details.
%
% You should have received a copy of the GNU General Public License
% along with thesis-manuscript.  If not, see <http://www.gnu.org/licenses/>.
%
% Created by: Lester James V. Miranda <ljvmiranda@gmail.com>
%=============================================================================

\chapter{Dataset Description}
\label{AppendixDataset}

\par The Yeast and Genbase datasets were from the works of
\cite{elisseeff2001kernel} and \cite{diplaris2005protein}.  Yeast contains
micro-array expression data and phylogenetic profiles with functions expressed
as MIPS\footnote{ Munich Information Center for Protein Sequences
(\cite{mewes2006mips}) } annotations, whereas Genbase consists of domain
characterizations of various proteins and their PROSITE IDs. Table
\ref{setup:datasets} details these two.

\begin{table}[!h]
    \centering
    \caption{Dataset description}
    \label{setup:datasets}
    \begin{tabular}{@{}r*{8}{l}@{}}
        \toprule
        Dataset & Instances & Features & Labels & Card.    & Dens.   & MaxIR     & MeanIR   & CVIR     \\ \midrule
        yeast   & $2417$    & $103$    & $14$   & $4.237$  & $0.303$ & $53.412$  & $7.197$  & $1.884$  \\
        genbase & $662$     & $1186$   & $27$   & $1.252$  & $0.046$ & $171.000$ & $37.315$ & $1.449$  \\ \bottomrule
    \end{tabular}
\end{table}

\par We can use various metrics to describe a multilabel dataset
(\cite{charte2015imbalance}). As a preliminary, we define an active label as a binary
$1$, while a non-active label as binary $0$. A sample $n$ with labelset
$\mathbf{y}_n = \left[1,0,1,0\right] (q=4, Y_n = \{\lambda_1, \lambda_3\})$ has
two active labels ($|Y_n|$). From this we can measure the following:
\begin{itemize}
    \item \textit{Cardinality}: the average number of active labels found per
        sample.
    \item \textit{Density}: amount of active labels with respect to all
        labels in the matrix.
    \item \textit{*IR}: statistical measurements of imbalance: max, mean, and
        variance.
\end{itemize}

\par Below are the equations for each metric:
\begin{align}
    \text{Card}(\mathcal{D}) &= \sum_{i=1}^{N}
    \dfrac{|{Y}_{i}|}{N} \\
    \text{Dens}(\mathcal{D}) &=
    \dfrac{\text{Card}(\mathcal{D})}{q} 
\end{align}

\par Data imbalance is one of the most common problems in machine
learning\textemdash especially when deploying models in the wild
(\cite{he2009learning}). Whenever one label dominates the other, the classifier
becomes severely affected: causing it to predict only the majority label. This
effect is felt in multilabel datasets, although not as pronounced in Yeast and
Genbase (\cite{ charte2015imbalance}).  In our work, we used a label weighing
scheme in the classifier to offset this problem (\cite{pedregosa2011scikit,
chang2011libsvm}). 

\newpage
\par To assess the severity of imbalance in a given dataset, we can measure 
the degree of imbalance for each label $\lambda$ and calculate the average for
all labels. \cite{charte2015imbalance} provides a good metric:
\begin{align}
    \text{IRLbl}(\lambda) =
    \dfrac{
        \text{argmax}_{\lambda^{\prime}=Y_1}^{Y_q}
        (\sum_{i=1}^{N} h(\lambda^{\prime},Y_i))
    }{
        \sum_{i=1}^{N} h(\lambda,Y_i)
    }, \quad h(\lambda, Y_i) =
    \begin{cases}
        1 & \lambda \in Y_i \\
        0 & \lambda \notin Y_i
    \end{cases} 
\end{align}

\par The equation looks a handful, but the intuition is to find the ratio
between the majority and minority labels. The value is $1.00$ for
$\lambda_{\text{major}}$ and higher for the rest. If we have $20$ labels
($q=20$), then we have twenty measurements for IRLbl. We can then calculate
their mean, max, and variance to obtain the MaxIR, MeanIR, and CVIR
respectively. Lastly, we can draw a co-occurence plot to visualize how
difficult these multilabel problems are. In Figure \ref{appendix:cooccurence},
each arc represents a label while each chord represents the frequency that two
labels occured simultaneously in a given sample.

\begin{figure}[!h]
  \centering
  \begin{subfigure}[b]{0.48\textwidth}
    \includegraphics[width=\textwidth]{appendix/yeast_labelset}
    \caption{Yeast}
    \label{cooccurence:yeast}
  \end{subfigure}
  \begin{subfigure}[b]{0.48\textwidth}
    \includegraphics[width=\textwidth]{appendix/genbase_labelset}
    \caption{Genbase}
    \label{cooccurence:genbase}
  \end{subfigure}
  \caption{Cooccurence plot for protein benchmarks}
  \label{appendix:cooccurence}
\end{figure}

\par The work on multilabel data in general is a good area for further research
(as addressed in Chapter \ref{ConclusionsChapter}). It deals with problems not
only in classification but also in data imbalance and the like. For more
information, we direct the reader to the works of \cite{charte2015imbalance,
charte2015mlsmote} and \cite{ charte2017dealing}
